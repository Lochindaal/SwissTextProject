\section{Experiments}

\subsection{Training Data}
To train our model we use a subset of the dataset provided by the \gls{stc} as well as the ArchiMob corpus. The \gls{stcd} contains 38 GB of labeled, 65 GB of unlabeled spoken swiss german audio data and an additional validation set
containing 1.5 GB of data \cite{stc2019}. The ArchiMob corpus (Release 2) contains X GB of spoken swiss german data \cite{archimob2016}. We remove all audio files, from the \gls{stcd}, where the quality of the translation
is rated lower than 0.7 (rating provided by \gls{stc}). Additionally, we also need to remove files larger than XY GB due to memory limitations.

\subsection{Evaluation metrics}
We evaluate our model through two types of metrics. The BLEU score \cite{Papineni2002BleuAM}, which is required by the \gls{stc} to compare the results. The \gls{stc} specifically requires a corpus
based BLEU score (todo cite stc), which aims at measuring the distance/similarity of the generated text and the provided ground truth.\\~\\Additionally, we measure the \gls{wer} for our models, as
this is a common metric for speech recognition systems \cite{Park2008AnEA}.

\subsection{Environment}
We train the model on a single NVIDIA GeForce RTX 2070 Laptop GPU (8 GB memory). We perform minimal tuning of our model's hyperparameters following the work of \newcite{Agarwal2020LTLUDEAL}.
.

\subsection{Models}
Following \newcite{pluss2020} we use a DeepSpeech architecture \cite{Hannun2014DeepSS} as our main model for speech-to-text translation. In order to get better results we use a
a pre-trained DeepSpeech model \cite{DeepSpeechGerman090} as the base model for most of our experiments. The first model we trained served as our private baseline. We trained a bare DeepSpeech model
on the labelled \gls{stcd}, which achieved a BLEU score of XY BLEU.
